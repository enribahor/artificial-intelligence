\documentclass[11pt, a4paper, spanish, openright, twoside]{book}
\usepackage[spanish, activeacute]{babel}
\usepackage[utf8]{inputenc}
%\usepackage[top=2.5cm, bottom=2.5cm, outer=1.75cm, inner=1.75cm, heightrounded, marginparwidth=2.5cm, marginparsep=0.3cm]{geometry}	%márgenes empequeñecidos
\usepackage[top=2.95cm, bottom=2.25cm, outer=2.75cm, inner=2.75cm, heightrounded, marginparwidth=2.5cm, marginparsep=0.3cm]{geometry}	%márgenes originalmente
\usepackage{dpg}
\usepackage{fli}

\usepackage{pgf}
\usepackage{tikz}

\usepgflibrary{shapes.geometric} % LATEX and plain TEX and pure pgf
\usetikzlibrary{arrows,automata,positioning}
\tikzstyle{accepting by double}= [double distance=1.6pt,double,outer sep=.5\pgflinewidth+.8pt] % esto es algo estético.
\renewcommand\shorthandsspanish{}  % para compatibilizar spanish con tikz

%%%%%%		Figuras		%%%%%%%%%%%%%%%%%%%
\usepackage[vflt]{floatflt}		%Entorno float-figure

%%%%%%		Page style		%%%%%%%%%%%%%%%%%%%
\renewcommand{\thepage}{\arabic{page}}% Arabic page numbers\fancyhead{}
\pagestyle{fancy}
\fancyfoot{}
\fancyhead[LO,RE]{Práctica 5}	%encabezado de pares: nombre de la sección
\fancyhead[RO,LE]{Diseño de un agente de viajes}
\fancyfoot[LE,RO]{\thepage}	%abajo a izqda en pares, derecha en impares: numero de pagina
%\fancyhead[LE]{\nouppercase{\leftmark}} %cuadro izquierdo de pagina par: parte y contador
\fancyfoot[CE]{Inteligencia Artificial} 
\fancyfoot[CO]{Doble Grado Informática-Matemáticas - Universidad Complutense}
\renewcommand{\footrulewidth}{0.4pt}
\renewcommand{\headrulewidth}{0.4pt}		% linea por debajo del encabezado
\renewcommand{\sectionmark}[1]{\markright{\textbf{\thesection. #1}}}	%negrita
\renewcommand{\labelitemi}{$\circ$} %Primer itemize con circunferencia vacia
\renewcommand{\labelitemii}{$\cdot$} %Segundo itemize con punto pequeño \cdot
\renewcommand*{\thesection}{\arabic{section}}	% Hace que no apareca el indice de capitulos y que comience en section

%%%%%%		Others		%%%%%%%%%%%%%%%%%%%
\setlength{\leftmarginii}{0em} %Segundo itemize sin sangria
\setlength{\leftmarginiii}{1em} %Tercer itemize casi sin sangria
\renewcommand{\labelitemiii}{ }
\pagenumbering{roman}
\addto{\captionsspanish}{\renewcommand*{\contentsname}{Índice}} %Cambia "Indice general" por "Indice"



\begin{document} 
\title{\Huge{\textsc{Inteligencia Artificial}} \\
	\vspace{0.7cm}
	 \textsc{\Large{Práctica 5}} \\
	\vspace{1.5cm}
	\includegraphics[scale=0.45]{viaje}}
\author{\textsc{Grupo 3:}\\
	Enrique Ballesteros Horcajo\\
	Ignacio Iker Prado Rujas}
\date{\Today}
\maketitle

\newpage
\mbox{}
\thispagestyle{empty}						% Hoja en blanco, sin numeros ni nada
\newpage


\tableofcontents 							%INDICE hipervinculado

\newpage
\mbox{}
\thispagestyle{empty}						% Hoja en blanco, sin numeros ni nada
\newpage

\pagenumbering{arabic}						% Pone el contador de paginas a 1 y ahora en numeros normales

\vspace{3cm}


\newpage



\begin{section}{Introducción: Diseño de un agente de viajes}

	\textbf{Problema:} 
	
	\textit{El objetivo de esta práctica es diseñar un agente de viajes que mediante la aplicación de los sistemas de reglas recomiende paquetes de viaje adecuados a los gustos y preferencias de los 
			clientes, tal y como haría un agente de viajes real.}
	
	\textit{Para ello debemos diseñar un prototipo que utilice el motor de inferencia de Jess y acotar el problema y el conocimiento necesario. Haremos un diseño realista, que podamos utilizar para implementar nuestro prototipo a partir de él. Por ello, 
		puede parecer que queda algo muy simple, pero el objetivo es implementar todo lo que se escriba en el diseño.}
	
	\textit{Nuestro prototipo pedirá una serie de datos al cliente y a partir de esos datos le recomendará una serie de paquetes de viajes ordenados en función de la coincidencia con sus gustos. Cada paquete 
		de viajes constará de un destino, un medio de transporte y un alojamiento.}

	
	\textit{Permitiremos el viaje en tren,  en avión, en autobús y en coche particular. No se podrán combinar los medios de transporte, por lo que a Canarias solo se podrá viajar a las ciudades con aeropuerto.}
	
	Si conseguimos implementar todo lo especificado a continuación, iremos ampliando el diseño de manera incremental.
	
	\begin{center}
	\includegraphics{espiral}
	\end{center}
	
\end{section}
	\newpage
	\begin{section}{Información sobre viajes disponibles y sus características}
		La información sobre los viajes debe ser información que pueda obtenerse a partir de bases de datos. No necesariamente de manera directa, pero debe poderse inferir de manera objetiva 
		a partir de datos que tengamos. La base de datos que vamos a utilizar será una propia creada a partir de los datos de \url{www.wikipedia.org}. 
		
		Tendremos solo destinos nacionales, restringidos a España, para centrarnos en ofrecer un turismo enfocado en satisfacer los deseos del cliente. Así, el coste del viaje pasa a un segundo plano.
		
		Los viajes saldrán todos desde Madrid.
		
		Sobre cada destino tendremos la siguiente información en nuestra base de datos:
			\begin{itemize}
				\item Nombre.
				\item Número de habitantes.
				\item Número de hoteles.
				\item Precio medio de un hotel (por noche).
				\item Número de albergues.
				\item Precio medio de un albergue.
				\item Número de campings.
				\item Precio medio de camping: por noche
				\item Número de monumentos oficiales.
				\item Montaña.
				\item Playa.
				\item Número de parques de atracciones.
				\item Número de discotecas.
				\item Medios de transporte permitidos: coche, avión, autobús o tren.
				\item Deporte que se pueden realizar: aventura, extremos, naturaleza.
				\item Número de balnearios. 
				\item Número de museos.
				\item Número de óperas.
				\item Número de restaurantes.
				\item Accesible para discapacitados.
				
						
			\end{itemize}
			
			\newpage
			Además, a partir de esta información inferiremos utilizando las siguientes reglas el resto de campos que utilizaremos para
			recomendar los viajes a nuestros clientes.
			
				\begin{itemize}
					\item Destino de ocio: Número de parques de atracciones $>$ 2 o Número de centros comerciales $>$ 5.
					\item Destino de cultura: Número de museos $>$ 5 o Número de teatros $>$ 5 o número de óperas $>$ 3.
					\item Destino de aventura: Existen deportes de aventura o existe playa o existe montaña.
					\item Destino extremos: Existen deportes extremos.
					\item Destino de fiesta: Número discotecas $>$ 10.
					\item Destino turístico: Número monumentos oficiales $>$ 10 o número de hoteles $>$ 100.
					\item Destino relax: Número de balnearios $>$ 0.
					\item Destino naturaleza: Existen deportes naturaleza o existe montaña.
					\item Destino gastronomía: Número de restaurantes $>$ 30.
					\item Fácil llegada: medio de transporte: avión, tren o autobús.
					
					
				\end{itemize}
		
	\end{section}
	
	\newpage
	\begin{section}{Información que es necesario obtener del usuario sobre sus necesidades, gustos y limitaciones (económicas, de tiempo, etc.).}
		Necesitaremos tanta información como sea posible del usuario, pero debe ser información precisa y objetiva. No podemos pedir que se 
		rellenen todos los campos. Habrá unos datos concretos sobre el cliente y unos datos variables sobre el viaje que desea consultar.
		Los datos del cliente serán:
		
				\begin{itemize}
					\item Nombre.
					\item Edad.									
				\end{itemize}
		
		Para cada viaje que quiera consultar deberá rellenar los siguientes campos obligatorios:
		
					\begin{itemize}
					\item Nº de personas.
					\item Nº de niños.
					\item Intereses en el viaje: relax, turismo, aventura, aventura extrema, gastronomía, ocio, cultura, fiesta y naturaleza. Deberá marcarse uno al menos.	
					
				\end{itemize}
		Además, se podrán rellenar los siguientes campos de manera opcional para recibir un servicio más personalizado.
		
				\begin{itemize}
					\item Discapacidad.
					\item Presupuesto por persona y día.	
				
				\end{itemize}
		Una vez rellenados los datos obligatorios y los que haya querido opcionales, inferiremos nuevos datos que almacenaremos en nuestro cliente:
		
				\begin{itemize}
					\item Intereses:
						\begin{itemize}
							\item Ocio si (Nº de niños $>$ 0)
							\item Fiesta si (Existe interés aventura y es joven)
							\item Cultura si (Existe interés turismo y es medio)
							\item Gastronomía si (Existe interés turismo y es medio)
							\item Fiesta si ( Existe interés en ocio y Nº de personas $>$ 1 y Nº niños $=$ 0)
							\item Turismo si (Existe interes cultura)
						\end{itemize}
					\item Rango de edad: 
						\begin{itemize}
							\item Joven si Edad $<$ 30.
							\item Medio si (Edad $>$ 29 y Edad $<$ 66).
							\item Mayor si Edad $>$ 65 
						\end{itemize}
					\item Presupuesto:
						\begin{itemize}
							\item Bajo: Menor de 30€.
							\item Medio: Entre 30€ y 50€.
							\item Alto: Más de 50€.
						\end{itemize}
						
				\end{itemize}
			
	\end{section}
		\newpage
		\begin{section}{Reglas de recomendación que pueden utilizarse.}
			Las reglas de recomendación deben relacionar la información que tenemos sobre los viajes y sus características con la información que tenemos 
			sobre el usuario.
			
			
			Usaremos los datos sobre el cliente para recomendarle tanto un alojamiento como un destino. 
			
			Así, recomendaremos destinos de la siguiente forma:
			
				\begin{itemize}

					\item Si existe interés en turismo y tiene edad media:
						\begin{itemize}
							\item Ciudad $\rightarrow$  Si ciudad es destino turístico.
						\end{itemize}
					\item Si existe interés en aventura y tiene edad media:						
						\begin{itemize}
							\item Ciudad $\rightarrow$  Si ciudad es destino de aventura.
						\end{itemize}
					\item Si existe interés en ocio y tiene edad joven y (Nº niños $=$ 0):
						\begin{itemize}
							\item Ciudad $\rightarrow$  Si ciudad es destino de ocio o si ciudad es destino fiesta.
						\end{itemize}
					\item Si existe interés en ocio y tiene edad media o mayor:
						\begin{itemize}
							\item Ciudad $\rightarrow$  Si ciudad es destino de ocio.
						\end{itemize}
					\item Si existe discapacidad:						
						\begin{itemize}
							\item Ciudad $\rightarrow$  Si ciudad está adaptada para discapacitados.
						\end{itemize}
					\item Si existe interés en fiesta y edad media o mayor:
						\begin{itemize}
							\item Ciudad $\rightarrow$  Si ciudad es destino de ocio.
						\end{itemize}
					\item Si existe interés en aventura extremay edad mayor:
						\begin{itemize}
							\item Ciudad $\rightarrow$  Si ciudad es destino aventura.
						\end{itemize}
					\item Si existe interés en turismo:
						\begin{itemize}
							\item Ciudad $\rightarrow$  Si ciudad es destino de turismo o destino de ocio.
						\end{itemize}
					\item Si existe interés en gastronomía:
						\begin{itemize}
							\item Ciudad $\rightarrow$  Si ciudad es destino de gastronomia.
						\end{itemize}
					\item Si Nº personas $>$ 4:
						\begin{itemize}
							\item Ciudad $\rightarrow$  Si ciudad es destino de ocio.
						\end{itemize}

				\end{itemize}
				
			\newpage
			A la hora de recomendar alojamiento, tendremos en cuenta el gasto diario por persona:
				\begin{itemize}
					\item Si gasto es bajo: Recomendamos camping.
					\item Si gasto es medio: Recomendamos albergue.
					\item Si gasto es alto: Recomendamos hotel.
				\end{itemize}
				
			Para el transporte también recomendaremos en función de los gustos(siempre que la ciudad lo permita):
				\begin{itemize}
					\item Si aventura, aventura extrema, ocio o naturaleza: Recomendamos coche.
					\item Si  turismo, fiesta o gastronomía y presupuesto bajo: Recomendamos autobús.
					\item Si turismo, fiesta o gastronomía y presupuesto medio: Recomendamos tren.
					\item Si relax: Recomendamos tren.
					\item Si presupuesto alto: Recomendamos avión.
				\end{itemize}
				
			
	\end{section}

	
\begin{thebibliography}{9}

\bibitem{aima}
	Russell, S.; Norvig, P, \\
	\emph{Artificial Intelligence, a modern aproach}.\\
	New Jersey: Pearson, 2010.
	
\bibitem{clase}
	Apuntes y transparencias de Inteligencia Artificial, \\
	Doble Grado Matemáticas - Ing. Informática, U.C.M., 2014-2015.

\end{thebibliography}


\end{document}