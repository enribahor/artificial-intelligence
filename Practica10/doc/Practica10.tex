\documentclass[11pt, a4paper, spanish, openright, twoside]{book}
\usepackage[spanish, activeacute]{babel}
\usepackage[utf8]{inputenc}
%\usepackage[top=2.5cm, bottom=2.5cm, outer=1.75cm, inner=1.75cm, heightrounded, marginparwidth=2.5cm, marginparsep=0.3cm]{geometry}	%márgenes empequeñecidos
\usepackage[top=2.95cm, bottom=2.25cm, outer=2.75cm, inner=2.75cm, heightrounded, marginparwidth=2.5cm, marginparsep=0.3cm]{geometry}	%márgenes originalmente
\usepackage{dpg}
\usepackage{fli}

\usepackage{pgf}
\usepackage{tikz}

\usepgflibrary{shapes.geometric} % LATEX and plain TEX and pure pgf
\usetikzlibrary{arrows,automata,positioning}
\tikzstyle{accepting by double}= [double distance=1.6pt,double,outer sep=.5\pgflinewidth+.8pt] % esto es algo estético.
\renewcommand\shorthandsspanish{}  % para compatibilizar spanish con tikz

%%%%%%		Figuras		%%%%%%%%%%%%%%%%%%%
\usepackage[vflt]{floatflt}		%Entorno float-figure

%%%%%%		Page style		%%%%%%%%%%%%%%%%%%%
\renewcommand{\thepage}{\arabic{page}}% Arabic page numbers\fancyhead{}
\pagestyle{fancy}
\fancyfoot{}
\fancyhead[LO,RE]{Práctica 10}	%encabezado de pares: nombre de la sección
\fancyhead[RO,LE]{Protégé con Jess}
\fancyfoot[LE,RO]{\thepage}	%abajo a izqda en pares, derecha en impares: numero de pagina
%\fancyhead[LE]{\nouppercase{\leftmark}} %cuadro izquierdo de pagina par: parte y contador
\fancyfoot[CE]{Inteligencia Artificial} 
\fancyfoot[CO]{Doble Grado Informática-Matemáticas - Universidad Complutense}
\renewcommand{\footrulewidth}{0.4pt}
\renewcommand{\headrulewidth}{0.4pt}		% linea por debajo del encabezado
\renewcommand{\sectionmark}[1]{\markright{\textbf{\thesection. #1}}}	%negrita
\renewcommand{\labelitemi}{$\circ$} %Primer itemize con circunferencia vacia
\renewcommand{\labelitemii}{$\cdot$} %Segundo itemize con punto pequeño \cdot
\renewcommand*{\thesection}{\arabic{section}}	% Hace que no apareca el indice de capitulos y que comience en section

%%%%%%		Others		%%%%%%%%%%%%%%%%%%%
\setlength{\leftmarginii}{0em} %Segundo itemize sin sangria
\setlength{\leftmarginiii}{1em} %Tercer itemize casi sin sangria
\renewcommand{\labelitemiii}{ }
\pagenumbering{roman}
\addto{\captionsspanish}{\renewcommand*{\contentsname}{Índice}} %Cambia "Indice general" por "Indice"



\begin{document} 
\title{\Huge{\textsc{Inteligencia Artificial}} \\
	\vspace{0.7cm}
	 \textsc{\Large{Práctica 10}} \\
	\vspace{1.5cm}
	\includegraphics[scale=0.45]{viaje}
	\includegraphics[scale=0.55]{protege}
	}
\author{\textsc{Grupo 3:}\\
	Enrique Ballesteros Horcajo\\
	Ignacio Iker Prado Rujas}
\date{\Today}
\maketitle

\newpage
\mbox{}
\thispagestyle{empty}						% Hoja en blanco, sin numeros ni nada
\newpage


\tableofcontents 							%INDICE hipervinculado

\newpage
\mbox{}
\thispagestyle{empty}						% Hoja en blanco, sin numeros ni nada
\newpage

\pagenumbering{arabic}						% Pone el contador de paginas a 1 y ahora en numeros normales

\vspace{3cm}


\newpage

\begin{section}{Descripción de la ontología desarrollada.}
	La ontología se basa en la práctica anterior desarrollada en Protégé. Se han añadido slots que no existían y que eran necesarios para Jess. Cada usuario tendrá una recomendación de viajes que puede 
	realizar, que le indicarán tanto el destino como el transporte y el alojamiento. Se mantienen los mayoristas que ofrecen viajes y las compañías hoteleras, aunque en la práctica no tienen gran uso.
	
\end{section}

\begin{section}{Programa Jess.}
	El programa en Jess automatiza el cálculo del presupuesto para los viajes basándose en el número de días y el coste del transporte y alojamiento. Además realiza el matching para recomendar al 
	usuario los viajes que puede realizar con su presupuesto y que debería hacer de acuerdo a sus gustos.
	 
\end{section}

\begin{section}{Objetivos pretendidos mediante la integración de Jess y Protégé.}
	Al integrar Jess y Protégé intentamos que ciertas partes de la ontología se rellenen de manera automática, permitiendo así la gestión de grandes datos de manera automática y personalizada para cada usuario. 
	Utilizando Protégé pretendemos que la organización de la información sea más intuitiva y que ciertos aspectos como la herencia sean más sencillos de gestionar. Jess gestiona ciertos aspectos automatizables que 
	dependen de diferentes clases, y el matching que rellena las recomendaciones en cada caso.
\end{section}

\begin{section}{Resultados obtenidos.}
	A pesar de las dificultades para trabajar en paralelo con Jess y Protégé, una vez resueltos los problemas los resultados son los esperados, obtenemos una lista de viajes recomendados para cada usuario, 
	y al obtenerla en Protégé podemos visualizar todas las características de cada viaje recomendado, y por supuesto del destino, transporte y alojamientos del mismo viaje. Mejora mucho la experiencia así respecto 
	a usar solamente Jess.
\end{section}

	
\begin{thebibliography}{9}

\bibitem{aima}
	Russell, S.; Norvig, P, \\
	\emph{Artificial Intelligence, a modern aproach}.\\
	New Jersey: Pearson, 2010.
	
\bibitem{clase}
	Apuntes y transparencias de Inteligencia Artificial, \\
	Doble Grado Matemáticas - Ing. Informática, U.C.M., 2014-2015.

\end{thebibliography}


\end{document}